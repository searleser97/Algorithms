\vspace{1em}
\sffamily

When performing operations with different primitive data types in C++,
the operand with smaller rank gets converted to the data type of the operand
with the greater rank.

\begin{center}
\begin{tabular}{| >{\centering}m{1cm}|m{7cm}|}
\hline
\multicolumn{2}{|c|}{\textbf{Data Type Ranks}}
\tabularnewline \hline
\textbf{Rank}
&
\centering
\textbf{Data Type}
\tabularnewline \hline
1
&
\begin{minipage}[c]{\linewidth}
\begin{minted}{cpp}
bool
\end{minted}
\end{minipage}
\tabularnewline \hline
2
&
\begin{minipage}[c]{\linewidth}
\begin{minted}{cpp}
char, signed char, unsigned char
\end{minted}
\end{minipage}
\tabularnewline \hline
3
&
\begin{minipage}[c]{\linewidth}
\begin{minted}{cpp}
short int, unsigned short int
\end{minted}
\end{minipage}
\tabularnewline \hline
4
&
\begin{minipage}[c]{\linewidth}
\begin{minted}{cpp}
int, unsigned int
\end{minted}
\end{minipage}
\tabularnewline \hline
5
&
\begin{minipage}[c]{\linewidth}
\begin{minted}{cpp}
long int, unsigned long int
\end{minted}
\end{minipage}
\tabularnewline \hline
6
&
\begin{minipage}[c]{\linewidth}
\begin{minted}{cpp}
long long int, unsigned long long int
\end{minted}
\end{minipage}
\tabularnewline \hline
7
&
{\fontfamily{lmtt}\selectfont\textcolor{pinegreen}{\textbf{\_\_int128\_t}}}
\tabularnewline \hline
8
&
\begin{minipage}[c]{\linewidth}
\begin{minted}{cpp}
float
\end{minted}
\end{minipage}
\tabularnewline \hline
9
&
\begin{minipage}[c]{\linewidth}
\begin{minted}{cpp}
double
\end{minted}
\end{minipage}
\tabularnewline \hline
10
&
\begin{minipage}[c]{\linewidth}
\begin{minted}{cpp}
long double
\end{minted}
\end{minipage}
\tabularnewline \hline
\end{tabular}
\end{center}
\vspace{1em}
Source: \textcolor{prussianblue}{https://en.cppreference.com/w/c/language/conversion}